\documentclass[fleqn]{article}

\usepackage{graphics}      % standard graphics specifications
\usepackage{graphicx}      % alternative graphics specifications
\usepackage{longtable}     % helps with long table options
\usepackage{url}           % for on-line citations
\usepackage{bm}            % special 'bold-math' package
\usepackage[OT2,T1]{fontenc}
\usepackage{fancyhdr}
\usepackage{amssymb}
\usepackage{color}
\usepackage{marvosym}   % astrological symbols
\usepackage{wasysym}    % astrological symbols
% \usepackage{mathabx}    % astronomical symbols
\usepackage{ccaption}
\usepackage{float}

\setlength{\topmargin}{-0.2in}
\setlength{\textwidth}{6in}
\setlength{\textheight}{8.5in}
\setlength{\oddsidemargin}{0.25in}
\setlength{\evensidemargin}{0.25in}
\setlength{\parskip}{2mm plus 1mm minus 0mm}
\raggedbottom

\newcommand{\lectitle}[1]{\begin{center} {\LARGE{\bf #1}} \end{center} \vspace{3mm}}

\newcommand{\mat}[2][rrrrrrrrrrrrrrrrrrrrrrrrrrrrrr]{
    \begin{array}{#1} #2\\ \end{array}}
\newcommand{\drawing}[1]{\fbox{\textcolor{red}{\parbox{4in}{Placeholder
	for the following figure: \\#1}}}\\}   
\renewcommand{\math}[1]{\begin{displaymath}#1\end{displaymath}}
\renewcommand{\labelitemii}{$\bullet$}
\renewcommand{\labelitemiii}{$\bullet$}
\newcommand{\ddx}{\frac{d}{dx}}
\newcommand{\ddy}{\frac{d}{dy}}
\newcommand{\ddt}{\frac{d}{dt}}
\newcommand{\ddr}{\frac{d}{dr}}
\newcommand{\del}{\nabla}
\newcommand{\pfrac}[2]{\frac{\partial #1}{\partial #2}}
\renewcommand{\sun}{\odot}
\renewcommand{\earth}{\oplus}
\newcommand{\half}{\frac{1}{2}}
\newcommand{\seq}{\hspace{1cm}=}
\renewcommand{\del}{\vec{\nabla}}
\def\farcs{\hbox{$.\!\!^{\prime\prime}$}}
\def\arcs{$^{\prime \prime}$}
\def\arc{$^{\prime}$}
\newcommand{\ds}{\displaystyle}
\newcommand{\cm}{\textrm{ cm}}
\newcommand{\s}{\textrm{ s}}
\newcommand{\erg}{\textrm{ ergs}}
\newcommand{\g}{\textrm{ g}}
\newcommand{\K}{\textrm{ K}}
\renewcommand{\c}{3\times10^{10} \cm/\s}
\newcommand{\AU}{1.5\times10^{13}\cm}
\renewcommand{\c}{3\times 10^{10}\cm/\s}
\newcommand{\gravG}{6.67\times 10^{-8} \cm^3/(\g \s^2)}
\renewcommand{\t}[1]{\textrm{ #1 }}

%\renewcommand{\figurename}{Figure}
%\captiondelim{. }
%\captionstyle{\\}

\newcommand{\fig}[3]{
\floatstyle{boxed} \restylefloat{figure} % This puts a box around the figure.
  \begin{figure}[!h]
    \centering
    \includegraphics[width=#3\textwidth]{#1}
%    \caption{#2}  % To produce numbered captions.
%    \abovelegendskip      
    \legend{{\em #2}} % to produce unnumbered captions
  \end{figure}
}
\newcommand{\figplain}[3]{
%\floatstyle{boxed} \restylefloat{figure} % This puts a box around the figure.
 \begin{figure}[!h]
    \centering
    \caption{#2}  % To produce numbered captions.
	      \includegraphics[width=#3\textwidth,angle=90]{#1}
 \end{figure}
}

\newcommand{\aside}[1]{
\shadowbox{\begin{minipage}{12cm}#1\end{minipage}}
}

% How to use subfigures:
% First, add \usepackage{subfigure} to the preamble.
% Then...
%
% \begin{figure}
% \centering
% \subfigure[Bild a.] % caption for subfigure a
% {
%    \label{fig:sub:a}
%    \includegraphics[width=4cm]{bild_a.eps}
% }
% \hspace{1cm}
% \subfigure[Bild b.] % caption for subfigure b
% {
%    \label{fig:sub:b}
%    \includegraphics[width=4cm]{bild_b.eps}
% }
% \caption{Bild a och b.}
% \label{fig:sub} % caption for the whole figure
% \end{figure}


\pagestyle{fancy}
\headheight 35pt

\rhead{G4J Manifesto}
\lhead{Last updated: \today}
\cfoot{\thepage}

\title{G4J Manifesto}
\begin{document}
\maketitle
%%%%%%%%%%%%%%%%%%%%%%%%%%%%%%%%%%%%%%%%%%%%%%%%%%%%%%
In the process of debugging my code, I've discovered that many (though
possibly not all) of the differences between my code's results and
Peter's stem from the way that the G4J values are calculated.

Equation (5.42) from Peter's cookbook [CITE] formulates the
discretized version of the radiative transfer equation as:
\begin{equation}
  T_{j+1} - T_{j} + \frac{(M_{j+1/2} - M_{j-1/2}) G M_j}{4\pi r_j^4}
  0.5(B_{j+1} + B_j) = 0
  \label{eq:eq1}
\end{equation}
where 
\begin{equation}
  B_j \equiv \frac{T_j}{P_j} \nabla
  \label{eq:eq2}
\end{equation}
Here, $\nabla$ represents the radiative/convective gradient value, and
is defined as
\begin{equation}
  \nabla = \pfrac{\ln P}{\ln T} = \frac{P}{T} \pfrac{T}{P}
  \label{eq:eq4}
\end{equation}

Equation (\ref{eq:eq1}) in its pre-discretized form reads as
\begin{equation}
  \frac{dT}{dM} = -\frac{GMT}{4\pi r^4 P}\nabla 
  \label{eq:eq3}
\end{equation}

This is, however, not the original form of the radiative transfer
equation.  Several assumptions and substitutions have transformed the
original expression
\begin{equation}
  \pfrac{T}{M} = -\frac{3}{64 \pi^2 ac} \frac{\kappa L}{r^4 T^3}
  \label{eq:eq4}
\end{equation}
to the form adopted by equation (\ref{eq:eq3}).  I will go through
those steps in detail, here.  
First,
\begin{equation}
  \pfrac{T}{M} = \pfrac{T}{P} \pfrac{P}{M} = \frac{T}{P} \pfrac{\ln
    P}{\ln T} \pfrac{P}{M} 
  \label{eq:eq5}
\end{equation}

Substituting (\ref{eq:eq4}) into (\ref{eq:eq5}) produces
\begin{equation}
  \pfrac{T}{M} = \frac{T}{P} \nabla \pfrac{P}{M}
  \label{eq:eq6}
\end{equation}

If (and only if!) the system is in hydrostatic equilibrium, the
following is true:
\begin{equation}
  \pfrac{P}{M} = -\frac{GM}{4\pi r^4}
  \label{eq:eq7}
\end{equation}

If the assumption of HSE holds, we could then combine
(\ref{eq:eq7}) and (\ref{eq:eq6}) to get
\begin{equation}
  \pfrac{T}{M} = - \frac{T}{P} \frac{GM}{4\pi r^4} \nabla
  \label{eq:eq8}
\end{equation}
which Peter's cookbook uses as the basis for his equation (5.42).

The misleading thing about (5.42), however, is that it assumes your
system is already in HSE.  If you're cold-starting a Henyey code
(e.g. from an n=3/2 polytropic model), that assumption is invalid.
Your initial model will not, necessarily, be anywhere near HSE.  The
Henyey code cannot assume initial HSE; that is what it's trying to
find, rather than what it's starting from.

This is reflected in Peter's Fortran code, if not in his equation
(5.42).  His code uses (\ref{eq:eq6}), rather than (\ref{eq:eq8}) to
calculate the G4J values (i.e., a quantitative measure how far out of radiative transport
equilibrium the model currently is).  
Because
\begin{equation}
  \pfrac{P}{M} \neq -\frac{GM}{4\pi r^4}
  \label{eq:eq9}
\end{equation}
when you're cold-starting a model, using (\ref{eq:eq8}) leads to
wildly wrong G4J values.  These incorrect G4J values, in turn, get
carried into the CDE calculations, and end up wreaking havoc with the
model's march towards convergence.


%%%%%%%%%%%%%%%%%%%%%%%%%%%%%%%%%%%%%%%%%%%%%%%%%%%%%%
\end{document}




%==================================================
% Reminders on how to use my OCW-style short-cuts
%==================================================

%\lectitle{Lecture/Paper title goes here}

%-----------------------------
% Example of how to insert a boxed figure

%\fig{diagram1}{}{0.4}
%-----------------------------

%-----------------------------
% Example of how to insert a ``lecture-style'' equation:
% i.e., left-justified, and NOT numbered

%\math{
%30^2 + x^2 = D^2
%}
%-----------------------------


%--------------------------------
% Example of how to do a table.

%\vspace{4mm}

%\begin{tabular}{l|l|l}
%x & y & accuracy\\
%\hline
%$x_o$ &$ 1$ & \\
%\hline
%$x_1$ &$ 2$ & \\
%\hline
%$x_2$ &$ \frac{7}{4}$ & \\
%\hline
%$x_3$ &$\frac{7}{4} +\frac{6}{7}$ & 1 part in $2\times 10^{-4}$\\
%\hline
%$x_4$ &$\frac{18,817}{10,864}$& 1 part in $2\times 10^{-9}$ 
%\end{tabular}

%\vspace{4mm}

%--------------------------------





